%%%%%%%%%%%%%%%%%%%%%%%%%%%%%%%%%%%%%%%%%%%%%%%%%%%%%%%%%%%%%%%%%%%%%%%%%%%%%%%
%
% witseiepaper-2005.tex
%
%                       Ken Nixon (12 October 2005)
%
%                       Sample Paper for ELEN417/455 2005
%
%%%%%%%%%%%%%%%%%%%%%%%%%%%%%%%%%%%%%%%%%%%%%%%%%%%%%%%%%%%%%%%%%%%%%%%%%%%%%%%%

\documentclass[11pt]{witseiepaper}

%
% All KJN's macros and goodies (some shameless borrowing from SPL)
\usepackage{KJN}
\usepackage[T1]{fontenc}
\usepackage{amsmath}
\usepackage{pgfgantt}
\usepackage{subcaption}
\usepackage{caption}
\usepackage{siunitx}
\usepackage{graphicx}
\graphicspath{{Images/}}
%
% PDF Info
%
\ifpdf
\pdfinfo{
/Title (Design Project)
/Author (Tristan Kuisis)
/CreationDate (D:201808300911)
/ModDate (D:200510121530)
/Subject (ELEN417/455 Paper Format, 2005)
/Keywords (Nothing)
}
\fi

%%%%%%%%%%%%%%%%%%%%%%%%%%%%%%%%%%%%%%%%%%%%%%%%%%%%%%%%%%%%%%%%%%%%%%%%%%%%%%%
\begin{document}

\title{Design Project}
\author{Tristan Kuisis
\thanks{School of Electrical \& Information Engineering, University of the
Witwatersrand, Private Bag 3, 2050, Johannesburg, South Africa}
}


%%%%%%%%%%%%%%%%%%%%%%%%%%%%%%%%%%%%%%%%%%%%%%%%%%%%%%%%%%%%%%%%%%%%%%%%%%%%%%%
%
\abstract{%The purpose of this document is to provide overview of the project undertaken from 16th of July to the 24th of August. 
Abstract Here}

\keywords{Keywords}


\maketitle
\thispagestyle{empty}\pagestyle{empty}


%%%%%%%%%%%%%%%%%%%%%%%%%%%%%%%%%%%%%%%%%%%%%%%%%%%%%%%%%%%%%%%%%%%%%%%%%%%%%%%
%




\section{INTRODUCTION} \label{sec:INTRODUCTION}

Over the last six decades, man has continuously propelled man made objects into space \cite{sputnik}. These objects have been said to exceed speed of $66 km/s$ \cite{fastestObject}, however, these extremely fast objects often leave the earths orbit and travel into outer space. Since the dawn of space exploration, many countries around the world have attemped, and in many cases, successfully deployed man made objects into the Earths upper atmosphere and into orbits around the Earth.

The organisations running these missions, in the early years, did not have sets of guidelines on how the missions should be carried out. In addition to this, there were many decades, during the Cold War era, that countries and organisations took extreme measures to make these missions a success. Many of the missions involved leaving many parts of the space craft in the Earths orbit \cite{spaceDebrisGuide}. Consequently, the number of debris orbiting the Earth grew over the years.
This was not thought to be an issue to the world at large as the belief was that space is immense and so there were no issues with using it as a dumping ground. It was not until the realisation by Kessler that the world began to worry \cite{Kessler}. This theory predicted that as the number of man made satellites (and other objects) in Earths orbit increases, so does the probability of collisions between them increase. When orbiting debris collide, the two objects can fragment (due to the excessive relative speeds with which the two objects travel at) and case multiple cascading collisions. This means that the debris orbiting the Earth would increase and result in greater difficulty for active space craft to undertake their missions.
Protecting active missions from this space debris is highly difficult as it is difficult to predict the density of objects in the specific orbit that the missions device should be in.

Guidelines were developed by NASA in the late 70's, however, this simply slowed the rate of introduction of debris and this did nothing to reduce the currently orbiting debris \cite{spaceDebrisGuide}. Recently, the United Nations General Assembly managed to get agreement between a number of countries to reduce the introduction of this debris \cite{debrisGuidelinesAgreement}.

There have been a number of methods in which to deal with this issue. The first is to protect the current missions from this debris, the first of these is to make use of a Whipple shield, this is simply an outer coating of space craft that is able to protect the craft against high velocity impacts of objects in outer space \cite{Whipple}. This shield is built to protect the craft from impacts from micrometeoroids in space.
This method falls short when the debris becomes large enough to penetrate the craft (and the Whipple shield), resulting in the destruction/damage of the craft.

A recent system is capable of removing debris from the Earths orbit, however, this is still in its early stages and has yet to prove highly effective \cite{removalSpaceDebris}.
The next method that systems currently make use of is object avoidance. This, however, is coupled with the fact that the location and trajectory of debris in orbit should be known and tracked. This forms the major task of this report.
Current solutions make use of either optical sensors which makes use of advanced telescopes in order to visually detect the debris. This system has a major drawback based on the fact that it can only be used during dawn and dusk hours \cite{OrbitalDebrisTechnicalAssessment,telescope,ZenithRanging}.
The second implementation makes use of radar techniques. This is the method which is implemented in this system.

This report begins with explaining the background to the task given, this includes an explanation of how the radar system works (in principle) it is then followed up with a description of the different ways in which the radar system can be implemented.
This is then followed up with a description of the space debris as well as a number of calculations on the characteristics of this debris.

Once the debris has been characterised, current implementations used for this application are analysed and compared.  

*** Need more here on the structure of the report.
% in the document: a-technical-assessment there is an image which breaks down all of the types of space debris


\subsection{Space Debris} \label{sec:SpaceDebris}
As discussed in section~\ref{sec:INTRODUCTION}, space debris in Earths orbit has been increasing for many decades, this is mainly due to space missions with poor regulations on the material that is allowed to stay in Earths orbit.
The majority of space debris currently is made up of man made air craft components, this includes objects such as: funcioning space craft, non-functional spacecraft, rocket bodies, exhaust products, objects created through deployment operations, and products of deteriorated space craft \cite{OrbitalDebrisTechnicalAssessment}.
Consequently, a large amount of this debris is made up of metallic materials as these materials make up the majority of the space crafts' components.
It has, in the past \cite{Spacex}, been difficult and costly to retrive the components which were used to get the space craft into orbit, this is a major contributer to the material.
This debris can occupy a number of regions within the Earths orbits and this changes its characteristics.
The current scope of the project is that it should be capable of tracking space debris that is within the Low Earth Orbit (LEO). This orbit is defined to be approximately $160 km$ to $2000 km$ altitude above the Earths mean sea level.
The second definition of these orbits is baed on orbital mechanics, this states that if an object is within an orbit, then it will have a corresponding velocity.
This is illustrated by equation~\ref{eqn:OrbitalVelocity}. 

\begin{equation} \label{eqn:OrbitalVelocity}
    V_{Object} = \sqrt{\frac{G M_{Earth}}{r_{Earth} + r_{Altitude}}}
\end{equation}
Where:
\begin{itemize}
    \item $V_{Object}$ is the radial velocity of the object
    \item $G$ is the gravitational constant ($6.67408 \cdot 10^{-11} m^3 kg^{-1} s^{-2}$)
    \item $M_{earth}$ is the mass of the Earth ($5972 \cdot 10^{24} kg$)
    \item $r_{Earth}$ is the mean radius of the Earth
    \item $r_{Altitude}$ is the altitude of the object above the mean radius of the Earth
\end{itemize}

This allows for the simple calculation of the minimum and maximum orbital speeds expected from the objects which correspond to the maximum altitude and minimum altitude respectively. These are illustrated in table~\ref{tab:ObjectParameters}. The velocity in the table is stated as a tangential velocity as it is assumed that the radial velocity of the object is zero. If an object has a radial velocity componenet, then this implies that it is changing orbits (an object in a specified orbit will have zero radial velocity) and this is either taking place due to a force acting on the object (man made) or it is experiencing a force due to the Earths atmosphere. The vectors which explain this movement of an object is illustrated in figure~\ref{fig:ObjectVelocityVectors}.

\begin{center}
    \begin{figure}
        \includegraphics[width=\textwidth]{Vectors.pdf}
        \caption{Object Velocity Vectors}
        \label{fig:ObjectVelocityVectors}    
    \end{figure}
\end{center}

The objects orbiting within these ranges have corresponding periods (amount of time it takes for an object to orbit around the Earth once), this means that an object in the lower orbit has a higher period than that of the higher orbits. The periods for the two limits of the LEO objects are also found in table~\ref{tab:ObjectParameters}.

Based on a number of physical limitations, the system implemented will only be capable of detecting the space debris once it appears within the field of view (FOV). The FOV is discussed in section~\ref{sec:FieldOfView}. Figure~\ref{fig:ObservableCharacteristics} illustrates the system and how its relation to the Earth. 
The inner circle represents the Earths surface and point P represents the position of the system on the Earths surface.  
The outer circule represents the path of an object orbiting the Earth.
The lenghts in this image are not to scale and do not represent the real situation, however, this is used for illustrative purposes.
It is highly important to determine the amount of time it takes from when the object first enters the systems FOV, until the point where it exits the systems point of view.
In this case it is assumed that the object travels directly over the boresite of the system and so it will take the maximum amount of time to travel over the field of fiew. In most cases, the objects do not travel directly over the FOV, and will therefore not be within the FOV for a reduced amount of time.
This illustrates the system in order to create a few starting assumptions/criteria for the system.
As can be seen in this image, the angle that is measured can be taken from two different reference points: the center of the Earth and the point P. Based on this, it can be seen that the angles that these two points create are different. This implies that if one were to calculate the maximum distance that an object can be detected (assuming a maximum FOV angle from point P), one cannot make use of the angle $\alpha$ and the distances of this arc as it would not represent the actual distances of the objects because the circle that the objects orbits around is different to the circle that is assumed the system makes use of.
Numbers $1$ and $3$ represent the points at which the object enters and leaves the FOV, and point $2$ represents the position of the systems zenith.

\begin{center}
    \begin{figure}
        \includegraphics[width=\textwidth]{ObservableCharacteristics.pdf}
        \caption{Observable Characteristics}
        \label{fig:ObservableCharacteristics}    
    \end{figure}
\end{center}


Based on the movement of objects in differing orbits, it can be seen that objects in higher orbits move slower than those in lower orbits.
The objects move with a specific velocity, as illustrated in table~\ref{tab:ObjectParameters}, this is measured in linear velocity, however, it is also important to represent the motion of these objects with the use of their angular velocity, the angular velocity is simply found with the use of equation~\ref{eqn:AngularVelocity}.

\begin{equation} \label{eqn:AngularVelocity}
\omega = \frac{v}{r}
\end{equation}
Where:
\begin{itemize}
    \item $\omega$ represents the angular velocity of the object (Radians per second)
    \item $v$ represents the linear velocity of the object
    \item $r$ represents the distance from the observer to the object
\end{itemize}


Once this has been estimated, it is important to evaluate the angle travelled by the object as seen from the observation point ($\alpha_{p}$). This is measured over a period of time ($\Delta T$) and the angular velocity calculated above ($\omega_{P}$) is used, this is found in equation~\ref{eqn:AngularVelocityP}

\begin{equation} \label{eqn:AngularVelocityP}
    \omega_{P} = \frac{\theta_{P}}{\Delta T}
\end{equation}
Where:
\begin{itemize}
    \item $\omega_{P}$ is the same as before
    \item $\theta_{P}$ represents the angle of the object seen from the observation point
    \item $\Delta T$ represents the amount of time that it takes the object to traverse the angle
\end{itemize}

This now allows for the calculation of the linear velocity of the object as seen from the observation point (it is the same if the center of the Earth is used as the reference point). This velocity is calculated with the use of equation~\ref{eqn:LinearVelocity}.

\begin{equation} \label{eqn:LinearVelocity}
    v = \omega_{P} h
\end{equation}
Where:
\begin{itemize}
    \item $v$ is the linear velocity of the object (m/s)
    \item $\omega_{P}$ represents the apparent angular velocity of the satellite (Radians)
    \item $h$ represents the altitude of the object
\end{itemize}


The use of the two equations above require the values for the amount of time that it takes for an object to enter the FOV of the system and exit it again. This is found with the use of equation~\ref{eqn:ObservableTime} \cite{ObservableTime}.

\begin{equation} \label{eqn:ObservableTime}
    t_{t} = \frac{r^{\frac{3}{2}}}{\sqrt{GM}} (\pi - 2E - 2 asin(\frac{R}{r} cos(E))))
\end{equation}
Where:
\begin{itemize}
    \item $t_{t}$ is the amount of time that an object is in the FOV of the system
    \item $r$ is the geocentric distance of the object (distance of the object above mean sea level)
    \item $G$ is the gravitational constant
    \item $M$ is the mass of the earth
    \item $E$ represents the elevation of the object above the horizon of the system
\end{itemize}

The elevation of the system is important as it relates to the FOV of the system. The two are interrelated by the fact that the elevation angle is the complementary angle of the maximum FOV of the system. This implies that if the elevation of the system is known, then the FOV of the system can be calculated.

In table~\ref{tab:ObjectParameters}, the elevation angle is assumed to be approximately $60^{\circ}$, this is a first estimate of the FOV of the system, further analysis of the system FOV of the system can be found in \ref{sec:FieldOfView}. This elevation angle creates a maximum amount of time that the object is visible. This implies that the object travels directly over the boresight of the system. In most circumstances, the objects will not travel directly over the boresight and will therefore be within the FOV for less time than stated in this table.

This creates the ability to calculate the observed velocity of the object as it passes over the FOV of the system, this is illustrated in figure~\ref{eqn:ObjectVelocity}.

\begin{equation} \label{eqn:ObjectVelocity}
    v = \omega_{P} h
\end{equation}
Where:
\begin{itemize}
    \item $v$ is the linear velocity of the object (m/s)
    \item $\omega_{P}$ represents the apparent angular velocity of the satellite (Radians)
    \item $h$ represents the altitude of the object
\end{itemize}


In addition to this, with an assumed elevation angle, the expected distances for the objects can be found, these are calculated making use of equation~\ref{eqn:ObjectDistances} \cite{ObservableTime}.


\begin{equation} \label{eqn:ObjectVelocity}
    \rho = R (\sqrt(\frac{r^2}{R^2} - (cos(E))^2) - sin(E))
    \end{equation}
Where:
\begin{itemize}
    \item $\rho$ is the slant distance from the system to the object
\end{itemize}

\begin{table}
    \begin{center}
        \begin{tabular}{ c c }
            \hline 
            Minimum Object Altitude & $m$ \\
            Maximum Object Altitude & $m$ \\
            Maximum Object Tangential Velocity & $m/s$ \\
            Minimum Object Tangential Velocity & b \\
            Maximum Angular Velocity & $m$ \\
            Minimum Angular Velocity & $m$ \\
            Elevation Angle (assumed for illustration) & $^{\circ}$ \\
            Minimum Observable Time & $s$ \\
            Maximum Observable Time & $s$ \\
        \end{tabular}
        \caption{Object Parameters}
        \label{tab:ObjectParameters}
    \end{center}
\end{table}

% What is an Antenna?
% What is an array?
% What is an antenna array?
% What is space?
% What is space debris? (How did it get there? why is it there? what is it made up of? where is it in space? how is it moving in space? is it bad/good? can we remove it? should we remove it, why? why do we care? who wants to know? fastest moving debris? how often do they come into contact with each other ? what happens when they come into contact with each other? can we protect against it? how is it currently being tracked?  )
% What are orbits?
% What is lower earth orbit? Geostationary orbit?
%  What is the atmosphere? (Range? made up of?)
%  What is the ionosphere? (Range? made up of? characteristics? EM characteristics? How fast can things move through them? )
%  current technology? radar and telescopes? advantages and disadvantages of these technologies?

The nature of an Incoherent Scatter Radar system is such that it directs electromagnetic energy into the earths "surrounding area"(ionosphere?); it highlights irregular characteristics present in this space. The energy that is transmitted is then reflected off of these irregularities and returns back in the direction of the system.
The system has the ability to create a narrow beam which transmits energy, this energy is then sterred (electronically) within the bounds of the system. Steering can be done in azimuth, elevation, and intensity. 


\subsection{Current Implementations}

\section{DEREK}

Radiation after the big bang (4 Kelvin). 
Noise floor. 

Specify a minimum size of objects: approx 10 cm (RCS (radio cross section))

Choose a receiver sensitivity: randomly: -120 dBm

The gain of the system needs to be determined as well as the transmitting power as you are required to receive at least -120 dBm of power back from an object of 10 cm

How narrow must your beam be in order to pick out individual elements floating around (this is a resolution issue)

Trade off between power and gain

Specify a frequency this can be because of monetary reasons as well as physical reasons.

Building an amp is a big mission. Lower frequency amps are easier to design/purchase.

Broadcast dipoles and how to make low VSWR versions

Degradation in the gain as you swing your major direction

Where will all of the side lobes go!?!?!? Will they affect other systems? Will they affect the readings and pick up objects that you dnot want?

Using the binomial you will reduce your sidelobes however you will reduce your overall gain.

What kind of beam sweep will you be looking for?

When it hits the object it will change polarization. We must check how many times if flips polarization and how it does this.

Maximum ratio combining (to do with the polarization things) This must be thought of, in how you combine these signals together. 

You need to provide time for "switches" to work, this is to give "dead time" for the system to set itself up.

How do you swap between the receiver and the transmitter.
This may be a case where a "[circulator](https://en.wikipedia.org/wiki/Circulator)" can be used.

Wilkinson splitter (for feeding the antenna)

The feeding network is different to how the antenna is connected together (Wilkinson splitter)

Hartebeespoort visit.

Where is the processing done? SHould the data be transmitted elsewhere and then processed? Then you won't need a server farm in the karoo and you can use the cloud.


Corporate feed

Determine the power that is present in the side lobes. 
I do not believe that it is worth using a binomial array, even though it reduces the side lobes, you still have to consider how the power system is structured. How do you distribute the amplifiers? Do you then have a case where you have a multitude of differently sized amplifiers in order to reduce the cost? 

Apply some form of taper for the amplitude over the area of the array.
In order to reduce the sidelobes, you must find a happy medium between gaussian distribution and a standard distribution.

Determine an acceptable dB level for the biggest sidelobes. Check for the power in the lobes and their direction (if they are pointing in dangerous areas or not). Another way to specify the first sidelobe level is to assume a maximum object size at the lowest altitude and then determine the gain for this sidelobe at which the signal is below acceptable levels.

Think about build up of static electricity within the system and consider how to protect circuitry from this (the receiver is commonly the most sensitive device)

It may make sense to apply a modulation to the signal as this can tell you if it is indeed your signal which you are detecting.



\section{CHOICE OF TECHNOLOGY}

\section{INDIVIDUAL AND ARRAY SIMULATIONS}

\subsection{System}
The system can be illustrated with the use of the design slides. This illustrates the entirety of the system.
The elements which make up the system include:

    * Transmitter
    * T/R Device
    * Antenna
    * Receiver
    * Signal Processor

Each of these items will have their own section.
However, there will be sections which discuss other components/characteristics of the system.

Section on EM waves and how they are created, influenced, propagated, polarized, phase shifted, their spectrums, superpositioned, 
Introduce the simple kinds of polarization and then reference later parts in the report that the waves will be affected by the structure of the antenna and the array as well as the atmosphere.

\subsection{Optimal Frequency}
[This](http://www.met.nps.edu/~psguest/EMEO\_online/module3/module\_3\_2b.html) seems useful

When determining the frequency, specify an opimal frequency, then provide optional operational frequency ranges (if possible). A discussion on how well the components work at these other frequencies and how the ionosphere handles these other frequencies, as well as the frequency bands that are available.
Speak about how the returned frequency will be different than the transmitted.

In choosing the frequency, it is highly advantageous to decide on a frequency where there is a decent amount of experience in the design and operation as well as it should be easy to find and purchase "off the shelf components". This implies that the frequency should lie near the EISCAT frequencies (224 and 931 MHz) or lie in the 400-500 MHz band which has been used by other incoherent scatter radars.

Higher frequency choices:

    * Lower system noise, this is highly dependent on the temperature above the site, these temperatures are different at different frequencies.
    * Smaller antenna size for a given beamwidth, apparently this makes it easier to achieve full steerability and a high scanning rate
    * Less clutter form coherent echoes which are mainly due to field-aligned irregulatities. The strength of these echoes will be about 12 times stronger at 430 MHz than at 931 MHz, and about 50 times stronger at 224 MHz.

Lower Frequency choices:

    * Smaller Debye ratio for a given electron concentraion so that measurements can be extended to greater heights
    * Greater cross-section for MST work

One final factor points to an intermediate frequency:

    * At higher frequencies the bandwidth of the scattered signal is greater. This widens the bandwidth over which noise is received and if the system noise temperature is approximately constant (i.e. for frequencies > 400 MHz) this increases the total noise power in proportion. However, with a broader bandwidth the maximum lag which must be measured in the autocorrelation function is reduced. This allows the envelope of the transmitted signal to be shorter and hence allows observations to begin at shorter range.

When using lower frequencies, the system noise limits the sensitivity massively, this also makes a narrow beamwidth which limits the E region (especially for observations near the zenith)

Another factor that affects the frequency band choice is that available by the systems regulator. A [document](https://www.icasa.org.za/uploads/files/ITU-Reference-Review-the-Radio-Frequency-Band-Plan-31264.pdf) provides the info for these bands.

ISM band (freely available, however everyone also is allowed to use it)

\subsubsection{Getting through the ionosphere}

It will be useful to take some info from [here](https://elib.dlr.de/110661/1/Initial\%20Detection\%20and\%20Tracking\%20of\%20Objects\%20in\%20Low\%20Earth\%20Orbit.pdf), on page 9 and before it talks about the right frequency windows that allow for transmission of EM waves.

Info from [here](http://www.met.nps.edu/~psguest/EMEO\_online/module3/module\_3\_2b.html)

There are limits to the frequencies that can be transmitted through the atmosphere+ionosphere. Below 5 MHz the radio waves are not transmitted because they are reflected by the ionosphere; above 30 GHz, the electromagnetic waves are absorbed by water vapour and carbon dioxide in the atmosphere.
[This](http://www.sws.bom.gov.au/Category/Educational/Other\%20Topics/Radio\%20Communication/Intro\%20to\%20HF\%20Radio.pdf) document goes through a number of calculations and information providing about the ionosphere and how the different frequencies are affected by it.
Speak about critical frequencies.
Docs: http://www.rfwireless-world.com/Terminology/Critical-Frequency-and-Maximum-Usable-Frequency.html
http://www.sws.bom.gov.au/Category/Educational/Other\%20Topics/Radio\%20Communication/Intro\%20to\%20HF\%20Radio.pdf
https://radiojove.gsfc.nasa.gov/education/educ/radio/tran-rec/exerc/iono.htm

Lower frequencies (less than 1GHz) result in reduced atmospheric losses, this is a major reason to choose this. Based on the textbook (Mark A. Richards, James A. Scheer, William A. Holm - Principles of Modern Radar Basic Principles, page 124, and 15), it is apparent that frequencies below the 1 GHz mark will have less than $10^-2 dB$ of attenuation per kilometer.

It is also stated that attenuation will decrease with an increasing elevation angle, and can also be neglected when the angle of elevation exceeds 10 degrees (textbook: Radar Essentials: A concise handbook for Radar Design and Performance Analysis G. Richard Curry, https://m.eet.com/media/1121840/912radar\_essentials\_pt1.pdf). 
The book also has some useful graphs which indicate the two way attenuation for 425 MHz at differing angles.
At the 10 degrees line, the two way attenuation does not exceed 0.25 dB depending on the range.

\subsubsection{ICASA}

It appears as if the last frequency allocation happened in 2013.
The next one is up for approval and is in draft stage. There are a number of bands which appear to be moving and this can affect things.

[This](https://www.icasa.org.za/legislation-and-regulations/national-radio-frequency-plan-2013) appears to contain the 2013 allocation.

Frequency allocation by [SKA](https://www.ska.ac.za/about/astronomy-geographic-advantage-act/)

The chosen frequency at this point is: 610 MHz.
The frequency on either side of the band is 4 MHz.
Wavelength: 0.4918 m
3/4 wavelength: 0.3689 m
1/2 wavelength: 0.2459 m
1/4 wavelength: 0.12295 m

subsubsection{Astronomy Geographic Advantage Act}

The minister has the ability to declare any area or part of an area in the Province of the Northern Cape as an astronomy advantage area. This implies that an area, which is technically advantagous to the functioning of the system can be demarkated by this act.
A number of proccesses are required to take place for these areas to be allocated to the system. These include a number of considerations for: local businesses, affected agricultural participants, environmental impacts, nearby electrical and radio interference.

The act allows for specific frequencies to be used (regardless of the current frequency allocation by ICASA) as long as it is assessed by the procedures required by the SKA. Permits can be issued by the act and they are then allowed to operate within the permitted areas. These permits include exemptions for the frequency spectrum that is allowed to be used and the permitted transmission characteristics.

The current situation with ICASA is such that there is a 50 mW maximum allowable broadcast power. However, with the use of the AGA, the TV transmission services are limited in the Northern Cape.

The band chosen is currently allocated ot the radion astronomy service on a primary basis.




\subsection{Field of View} \label{sec:FieldOfView}



The radar equation forms the initial stage for the design process.

Primary functions:

    * Impedance transformation (free-space fields to guided waves)
    * Propagation-mode adapter (free-space to guided waves)
    * Spatial filter (radiation pattern - direction-dependent sensitivity)
    * Polariation filter (polarization-dependent sensitivity)

The antenna is used to fix the mismatch between free-space and the rest of the system.
Intrinsic impedance of free-space: $n_0 = E/H$
$n_0 = sqrt(u_0 / e_0) = 120 * pi ~= 376.7 \Omega$

Characteristic impedance of tx line, $Z_0 = V/I$ (typically $50 \Omega$)

\section{Polarization}

Make use of [this](https://etd.ohiolink.edu/!etd.send\_file?accession=ohiou1345227397\&disposition=inline) to speak about the axial ratio and how polarization is done.

\subsection{Array Structure}

\subsubsection{Monostatic versus Bistatic}

There are two major radar configurations that are used. 

\subsubsection{Bistatic}

The bistatic configuration makes use of a separate transmitter and receiver for the system. The two components are placed at differing locations and the separation between the components is required to be sufficiently large in terms of the angles or ranges that they present.
This implies that the two systems, in the case of Earth-to-Space transmission, are required to be geographically (in the order of kilometers) far apart \cite{bistaticNato}.
The transmitters in these systems generally transmit power in the range of kilowatts to megawatts, this is in order to increase the signal-to-noise (SNR) of the system as there is a significant amount of energy lost in the energy transfer process. The receiver on the other hand, works with milliwatts to nanowatts as it receives this greatly reduced power that is reflected from the objects in question. The bistatic system provides a concenient method of separating out the two systems such that it becomes increasingly difficult to damage the receivers equipment from the high power generated by the transmitter.

\subsubsection{Monostatic}

The other, more commonly used system, is the monostatic configuration. This bundles the transmitter and receiver together on the same antenna/antenna array. In this case, a system is required such that the transmitted and received signals are isolated from eachother when they interact with the system. The isolation can be carried out in different ways, a circulator or a switch commonly achieves this. These components are discussed in a later section ***. In the monostatic configuration, the transmitter and receiver do not operate at the same time which greatly simplifies the switching apparatus.

\subsubsection{Comparison}

The configuration implemented is the monostatic as it provides a lower cost alternative to the bistatic configuration. It has the added benefit of only requiring one piece of land for the system. The system cost is also reduced by the fact that only one set of antennas are required.
The bistatic configuration is easier to implement with regards to the protection systems for the receiver, however, this benefit does not out weigh the monostatic systems advantages. The bistatic systems also require a significant amount of communication between the two systems and this can increase the cost of implementation, most notably when the distance between the two systems is large and communication mediums can become prohibitively expensive.

\subsection{Continuous Wave and Pulsed Wave}

The two commonly used transmission systems are the continuous wave and the pulsed wave, these two systems are also linked to the choice on the system configuration.
This implies that if the monostatic configuration is used, then the continuous wave transmission system is not available.

\subsubsection{Continuous Wave Transmission}

The continuous wave transmission system, as the name indicates, has the transmitter functioning $100\%$ of the time, which implies that the receiver also functions all of the time. The continuous wave transmission system, as indicated above can only be used by the bistatic configuration. 
A unique characteristic of the continuous wave system is that it is unable to determine the electromagnetic (EM) waves round-trip time as there is no set begin and end time of transmission. This can be solved with the use of modulation on the wave, this implies that the receiver is required to know all of the characteristics of the transmitters waveforms ahead of time if it is to determine round trip times \cite[p.~20]{radarHandbook}.

\subsubsection{Pulsed Waveform Transmission}

The second mehtod of EM wave transmission makes use of pulses, these occur periodically and their duration is commonly over a short period of time. This system is commonly used in tandem with a monostatic configuration and this provides the isolation between the transmitter and receiver circuitry, the circuitry to isolate these signals is still used.
Following the EM wave pulse, the receiver is set to record the resulting EM waves that are returned in the form of echoes after it has reflected from the objects in question. The determination of the pulse length and the period between pulses (the interpulse period (IPP))is discussed further in the report ***. The pulse repitition is also highly pertinent to the observable time window of debris, this will be discussed in section ***.

Need image here which describes how the pulses are placed with respect to time.

The pulse repetition frequency is related to the maximum distance that the system is able to detect objects, this is related to the observable area of the system and the furthest distance that the system can detect an object. This is discussed in section *** and section ***.

The following image represents a case where a pulse is sent outwards and is then followed by, a period of time later, a corresponding echo pulse.
The period between the pulse and the echo can be used to determine the altitude that the object has.
In order to detect the pulse, the time between each sample should be no less than the pulse width of the transmitter (this should be elaborated on... get the info from: \cite[p.~21]{radarHandbook})

Think about echoes that come back after the next pulse is fired, this can be from targets that are past the 2000 km range. These need to be dealt with. One way is to define a SNR that will rule out these cases. Another case is to throw out any echoes that happen before one is expected to have returned (based on the minimum altitude required to be detected). 
This issue is known as a range ambituite (\cite[p~.22]{radarHandbook})

*** Include calculation of the pulse width and the corresponding bandwidth if requires. Stuffies for this can be found on page 65

\subsection{Threshold Detection}

In order to detect a signal, the received signals' power is required to be above a specific threshold. This threshold is defined as the level at which all sources of noise are incorporated. When an object is detected, the signal level is required to rise above this threshold and this is then considered a successful detection.

There are cases, based on the fact that noise is inherently a random variable, that the noise level can exceed the threshold set by the system. This is observed as a false alarm.
This implies that there are cases where the targets echo in addition to the noise are capable of being below the threshold, this is when the noise in the system is low, so the addition of the echo is not large enough for the system to detect the object.
These two cases are highly unlikely, however, cannot be ignored as they are based on the probabilities of the entire system.

These probabilities are given as two variables: the probability of detection ($P_{D}$), and the probability of false alarm ($P_{FA}$). The probability of detection is the probability that an object-plus-noise exceeds the threshold applied to the system. The probability of false alarm is the probability that the noise (from all components of the system) will exceed the threshold.
The best case scenario for these two variables are: 
$P_{D} = 1$
and
$P_{FA} = 0$

These values are idealistic and so the correct threshold value is required to be set such that the system can operate as close to these values as possible.
When the threshold is increased the probability of false alarm will decrease, however, so will the probability of detection. When the threshold is decreased, the probabilities of both variables increase. A trade-off is required here to determine the optimum threshold level.

There is one way in which to increase the probability of detection while lowering the probability of false alarm, this is achieved by increasing the targets signal power (it should be noted that the signal power should be increased relative to the noise power). 
<!-- If the noise power increases in ratio with the overall power, then this will not work -->

The importance of Doppler should not be underestimated. Doppler shift is important as it can be used to reduce the effects of clutter in the viewing area, it is also able to perform measurements to detect if there is more than one object in the viewing area (in the same range). The removal of clutter from view of the system is achieved with Doppler by the fact that clutter is not moving and thus will have a no frequency shift.

\subsection{Resolution}

The resolution of the system has three components: the ability to distinguish between objects that are different distances away from the system, have differing angles with respecto the the system, and have differing Doppler frequencies.

The first of these, the range resolution, determines if you are able to discren two objects that are in eachothers vacinity.
The range resolution is defined by:
$\delta R = \frac{c \tau}{2}$

This implies that if the pulse width of the signal is 1us, then the range resolution of the system is $150 m$. This implies that if the pulse width is decreased, then the range resolution increases.
It must be noted that, if the pulse width is decreased, then the energy that is transmitted will decrease, and therefore it becomes increasingly difficult to detect an object.

% The doppler resolution is determined by: 3 dB width = \frac{0.89}{dwell time}

\subsection{Radar Functions}

\subsubsection{Search/Detect}

\subsubsection{Track}

\subsubsection{Image}

\subsection{SNR}

The resulting SNR of the calculation performed is the targets signal power divided by the sum of the noises in the system, this includes the receiver thermal noise and jammer noise.

The set of equations that are used to determine a number of these parameters are known as the radar range equations (RRE), these equations allow for the determination of the received power from the systems EM waves' reflections and also to determine the levels of the interfering power within the system, this allows for the calculation of the SNR.

This section begins with introducing the RRE and how it is applied to the system and how to determine the simplistic performance of the system. This first step estimates the performance of the systems power density at a distance R away. This is followed by estimating the thermal noise present in the receiver of the system, this then allows for the first estimate of the SNR. 

\subsection{Power Density}

The first equation defines the power that is incident on the target, this is given by equation ***

$P_{inc} = P_t * \frac{G_{t}}{4 * pi * R^2} [W/m^2]$

Where:
\begin{itemize}
    \item $P_{inc}$ is the power incident on the target
    \item $P_t$ is the power transmitted from the system
    \item $G_t$ is the gain of the system
    \item $R$ is the distance from the system to the target
\end{itemize}

This equation assumes the use of a directional antenna (not isotropic) which has a corresponding gain. This allows for the concentration of the beam in a specific direction.

The next stage is to determine the magnitude of energy that is reflected by the target, as assumed in section ***, the radar cross section of the target is defined and known. 
In depth analysis of the targets is provided in section ***.
Equation *** indicates the magnitude of power reflected by this object.

$P_{refl} = \frac{P_{t} G_{t} \sigma}{4 pi R^2}$

Where:
\begin{itemize}
    \item $P_{refl}$ is the reflected power from the target
    \item $\sigma$ is the radar cross section (RCS) of the target measured in square meters ($m^2$)
\end{itemize}

This reflected power from the target is then received by the system and this is defined by equation ***

$P_{r} = \frac{P_{t} G_{t} G_{r} \lambda^2 \sigma}{(4 pi )^3 R^4}$

Where:
\begin{itemize}
    \item $P_{r}$ is the received power at the system
    \item $A_{e}$ is the effective aperture of the system
    \item $\lambda$ is the wavelength of the system
    \item $G_{r}$ is the gain of the receive antenna
\end{itemize}

It is assumed that the gain is determined with equation ***.

$G = \frac{4 pi n_{a} A}{\lambda^2} = \frac{4 pi A_{e}}{\lambda^2}$

Where:
\begin{itemize}
    \item $n_{a}$ is the efficiency of the antenna created (this is commonly between 0.5 and 0.8 \cite[p.~64]{radarHandbook})
\end{itemize}

In the case of the monostatic antenna array, the transmitting and receiving gain are the same as it is the same array. This implies that for bistatic systems, it is possible to have two different ranges for receiving and transmitting.


\subsubsection{Receiver Noise}

\subsubsection{Thermal Noise}

The noise from the environment is mostly made up by solar effects. This case deals specifically with the noise generated by the sun. This can contribute to large amounts of noise within the signal if the antenna array is pointed in the direction of the sun, however, great care should be taken to avoid pointing the antenna array directly at the sun. There will still be a small contribution of the suns noise from the sidelobes of the antenna array, however, this is greatly reduced.

The second source of noise, the largest component within the system, is that generated by the receiver electronics.
Thermal noise power, being uniform over the frequency spectrum contributes to the noise power over the bandwidth of the system. This implies that the thermal noise power is directly proportional to the receivers bandwidth. This power is determined with the use of equation ***.

$P_{n} = k T_{s} B = k T_{0} F B$

Where:
\begin{itemize}
    \item $k$ is Boltzmann's constant ($1.38 \cdot 10^{-23} Watt-sec/K$)
    \item $T_{0}$ is the standard temperature (normally $290 K$)
    \item $T_{s}$ is the system noise temperature ($T_{s} = T{0} F$)
    \item $B$ is the receivers bandwidth ($Hz$)
    \item $F$ is the noise figure of the receiver subsystem.
\end{itemize}

The noise figure defined here should be given in real amplitude (convert from $dB$ as this is what it is usually specified in).
The bandwidth defined here is a calculated value and is dependent on two factors of the system.
The first of these can be defined with the use of the largest doppler frequency shift detectable by the system. This is defined in *** with the introduction of the debris' characteristics.
The second of these is defined by the minimum pulse width of the signal. This is calculated in section ***. 
This second case is highly important because if the bandwidth of the receiver is made smaller than the bandwidth required for the pulse width, then the power on the target will be reduced and this will cause inconsistencies and will reduce the range resolution of the system \cite[p.~65]{radarHandbook}. If the bandwidth of the receiver is created to be larger than the reciprocal of the pulse width, then the SNR of the system will decrease. This implies that the bandwidth of the receiver can occupy a small frequency band which is set by the pulse width. This implies that if the pulse width of the system is set to 1us, then the bandwidth of the system is 1 MHz.
The approximation of using $1/\tau$ is often used in monostatic systems \cite[p.~65]{radarHandbook}.

\subsection{SNR and the RRE}

At this point it is possible to evaluate the SNR of the system with the use of the noise figure that is attained above.
The SNR is defined by equation ***.

$SNR = \frac{P_{t} G_{t} G_{r} \lambda^2 \sigma}{(4 pi)^3 R^4 k T_0 F B}$

This equation applies to discreet targets, which applies to this circumstance.

\subsubsection{Multi-Pulsing}

Need to talk about how the SNR can be improved with the use of multiple pulses.

\subsection{Directions}

Discuss how the system cannot determine the direction of the object, it can only determine if the object is moving towards the antenna's boresight or away (with an increased doppler freq or a decreased doppler freq)


\section{Environmental Considerations}

\subsection{Weather Dependence}

Maximum and minimum temperatures. Average temperatures, temperature swings and their gradients.
Wind.
Fog
Sun angles

Light intensities

Air quality
Humidity
Icing of instruments
Snow


Light polution



\subsection{Land}

[This](https://www.icasa.org.za/legislation-and-regulations/protection-of-karoo-central-astronomy-advantage-areas-regulations-2017) will be useful for talking about how the area will be protected.
Land can be given by the [SKA](https://www.ska.ac.za/about/astronomy-geographic-advantage-act/). One can argue that the system will not be used all the time, as such, the SKA may be able to make use of the system for research purposes.
The lay of the land
Mountains?
Flat land?
How difficult will it be to lay foundations
Shipping material
Construction side of things
Altitude? Height above sea level?
Nearby roads?
Quality of these roads? Will they need upgrading? 
Power distribution? 
Communications?
Water?
Sewage?
Disturbances of power lines and transformers to the system?
Frequency of the power?

Lightning?

Nearby electromagnetic interference?
Can this interference be removed? Must it be quantified and then subtracted from the overall signal? Is this sufficient?

Possible impact from telecommunication in the area. Do radio links exist? 
One must look at the sidelobes and the distribution of power in the various harmonics

Air traffic interference?
Airports? planes overhead?
Mostly planes? Helicopters? They pose different interferences as they fly at different altitudes and use different communication systems.

In the design paper, they say that helicopters frequent the area, however, on days of good visibility they fly low and follow the valleys where they will be shielded against the radar beam.

It may be worth installing a separate small surveillance radar that is capable of detecting aircraft and then switch off the main radar (depending on the situation)

Is it worth contacting the air ports and the control towers for some of this? Get real time tracking of planes?

Beacons in the area?

Displacement of plants and animals?

Elevation of the horizon? Does it change in different directions? Does this affect the system?


\section{Construction}

How things fit together, how the cables are placed
What kind of cables
What communications medium is used between modules



\section{COST OPTIMISATION}

\section{CHOICE OF PHYSICAL LOCATION IN SOUTH AFRICA}

\section{WHAT IMPACT WILL THE SYSTEM HAVE ON THE ENVIRONMENT}

\section{SENSITIVITY ANALYSIS}

\section*{ACKNOWLEDGEMENT} \label{sec:ACKNOWLEDGEMENT}


%%%%%%%%%%%%%%%%%%%%%%%%%%%%%%%%%%%%%%%%%%%%%%%%%%%%%%%%%%%%%%%%%%%%%%%%%%%%%%%
%
%\nocite{*}
\bibliographystyle{witseie}
\bibliography{references}


%{\tiny \vfill \hfill \today \hspace{5mm} witseie-paper-2003.\TeX}

\end{document}

" vim: ts=4
" vim: tw=78
" vim: autoindent
" vim: shiftwidth=4
